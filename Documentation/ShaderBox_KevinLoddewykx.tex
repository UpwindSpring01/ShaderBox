\documentclass{paper}
\usepackage[outputdir=temp]{minted} % code formatting
\usepackage[svgnames, dvipsnames, table]{xcolor}
\usepackage{graphicx}
\usepackage{epspdfconversion} % Support for eps graphics
\usepackage{fancyhdr}
\usepackage[color]{changebar} % for the vertical line on frontpage
\usepackage{hyperref} % url and url coloring support
\usepackage[export]{adjustbox}
\usepackage[utf8]{inputenc} % allows accented words
\usepackage[T1]{fontenc} % these two following packages make copy & paste and search of accented words possible
\usepackage{lmodern}
\usepackage{tocbibind} % make the bibliography appear in Table of Content
\usepackage{parskip} % add gap after \par
\usepackage[breakable]{tcolorbox} % needed for box which holds the code
\usepackage{accsupp} % Used by minted internally, needed for disallowing copy paste of linenumbers, headers and footers
\usepackage{pifont} % allow for printing special symbols
\usepackage{libertine} % font package used for quote marks
\usepackage{tikz}
\usepackage{enumitem} % adds support of noitemsep to itemize
\usepackage[a4paper]{geometry} % change margins

\newcommand*\quotefont{\fontfamily{LinuxLibertineT-LF}} % selects Libertine as the quote font

% Make line numbers, headers and footers non-copyable, works with Adobe Reader.
% Fails in Edge, Chrome and Reader
\newcommand\emptyaccsupp[1]{\BeginAccSupp{ActualText={}}#1\EndAccSupp{}}
\let\theHFancyVerbLine\theFancyVerbLine
\def\theFancyVerbLine{\rmfamily\tiny\emptyaccsupp{\arabic{FancyVerbLine}}}

\tcbuselibrary{minted, skins}
\usemintedstyle{vs}
\setminted
{fontsize=\footnotesize,
	linenos,
	breaklines=true,
	breaksymbolleft=,
	tabsize=4,
	numbersep=3mm,
}

\tcbset{listing engine=minted,
	colback=LightGrey,
	colframe=black!70,
	listing only,
	left=5mm,
	enhanced,
	breakable,
	overlay={\begin{tcbclipinterior}\fill[black!25] (frame.south west) rectangle ([xshift=5mm]frame.north west);\end{tcbclipinterior}}]
}

\hypersetup{colorlinks = true, urlcolor = blue, linkcolor = blue, citecolor=blue}
\newcommand{\frontpage}
{
	\begin{titlepage}
	
	\raggedleft
	\changebar
	\vspace{\baselineskip}

	{\small Kevin Loddewykx}\\[0.167\textheight]
	{\large \bfseries Documentation v1.0}\\[\baselineskip]
	{\textcolor{Red}{\huge Shader Box}}\\[\baselineskip]
	{\large \textit{A HLSL library and Editor Tool}}\\[\baselineskip]
	%\includegraphics[scale=0.5]{unity-master-black}
	\vfill
	%{\includegraphics[scale=0.5]{Logos/unity-master-black}}
	\endchangebar
	\end{titlepage}
}

\newlength{\imgwidth} % create new variable for storing image width, used in the command declared below
\newcommand\scalegraphics[3][]
{
	\begin{figure}[H]
	\centering
	\settowidth{\imgwidth}{\includegraphics{images/#2}} % calculate image width and store in imgwidth
	\setlength{\imgwidth}{\minof{#1\imgwidth}{\textwidth}} % take minimum of imgwidth scaled by the optional parameter and textwidth
	\includegraphics[width=\imgwidth]{images/#2} % place graphics
	\caption{#3}
	\end{figure}
}

\newcommand*\bracket[1]{\lbrack#1\rbrack} % command to place the parameter between square brackets

\newcommand*\quotesize{60} % if quote size changes, need a way to make shifts relative
% Make commands for the quotes
\newcommand*{\openquote}
{
	\tikz[remember picture,overlay,xshift=-4ex,yshift=-2.5ex]
	\node (OQ) {\quotefont\fontsize{\quotesize}{\quotesize}\selectfont``};\kern0pt
}

\newcommand*{\closequote}[1]
{
	\tikz[remember picture,overlay,xshift=4ex,yshift={#1}]
	\node (CQ) {\quotefont\fontsize{\quotesize}{\quotesize}\selectfont''};
}

% select a colour for the shading
\colorlet{shadecolor}{Azure}

\newcommand*\shadedauthorformat{\emph} % define format for the author argument

% Now a command to allow left, right and centre alignment of the author
\newcommand*\authoralign[1]
{%
  \if#1l
    \def\authorfill{}\def\quotefill{\hfill}
  \else
    \if#1r
      \def\authorfill{\hfill}\def\quotefill{}
    \else
      \if#1c
        \gdef\authorfill{\hfill}\def\quotefill{\hfill}
      \else\typeout{Invalid option}
      \fi
    \fi
  \fi
}
% wrap everything in its own environment which takes one argument (author) and one optional argument
% specifying the alignment [l, r or c]
%
\newenvironment{shadequote}[2][l]%
{
	\authoralign{#1}
	\ifblank{#2}
	   {\def\shadequoteauthor{}\def\yshift{-2ex}\def\quotefill{\hfill}}
	   {\def\shadequoteauthor{\par\authorfill\shadedauthorformat{#2}}\def\yshift{2ex}}
	\begin{snugshade}\begin{quote}\openquote
}
{
	\shadequoteauthor\quotefill\closequote{\yshift}\end{quote}\end{snugshade}
}

\pagestyle{fancy}
\fancyhead{}
\fancyhead[R]{\emptyaccsupp{Unity\textregistered\ Framework}}
\fancyfoot{}
\fancyfoot[R]{\emptyaccsupp{\thepage}}
\fancyfoot[L]{\emptyaccsupp{Section \thesection}}
\fancyfoot[C]{\emptyaccsupp{Kevin Loddewykx}}

\begin{document}
\frontpage

\newpage
\tableofcontents

\raggedright

\newpage
\section{ShaderBox}
Shader Box is an open-source tool that allows you to create HLSL shaders for the DirectX api, and is currently still a prototype. It supports the shader stages: Vertex, Hull, Domain, Geometry and Pixel. The UI is written in WPF with C\#, and the engine which powers the rendering of the 3D viewport is custom written in C++. The application makes use of the Model-View-ViewModel (MVVM) pattern.
\par
I, Kevin Loddewykx, started this project as a school project for the course Tool Development, where we were tasked to create a WPF tool, at Howest: \href{http://www.digitalartsandentertainment.com/}{Digital Arts and Entertainment}. The project aims to be a replacement for \href{https://developer.nvidia.com/fx-composer}{FX Composer}, which is no longer in development and only supports shaders up to DirectX 10 which doesn't include the Tessellation pipeline stage.
\par
The UI is written in WPF with C\#, and the engine which powers the rendering of the 3D viewport is custom written in C++. The application makes use of the Model-View-ViewModel (MVVM) pattern. The parsing and lexing of HLSL code is done by an external C\# library namely HLSL Tools.
\par
\begin{itemize}[noitemsep]
\item Forward rendering
\item Post Processing
\item Vertex, Hull, Domain, Geometry and Pixel stages
\item Toplogies: Triangle List, Triangle List with Adjacency, Patch List with 3 control points
\item Cull modes: back, front and none
\item Fill modes: solid and wireframe
\item Include directives
\item Wavefront .obj models
\item Textures
\item Properties panel for tweaking shader parameters
\item Undo/Redo support
\item snorm and unorm support
\end{itemize}

\par
The app has three tabs each with their own subpanels: Library \bracket{\ref{subsec:tab_l}}, Editor \bracket{\ref{subsec:tab_e}}, Data \bracket{\ref{subsec:tab_d}}.

\newpage
\subsection{Planning}
Currently I am working on creating a UI design document. Describing the future UI layout, taking into account al the planned and backlogged feautures and extensibility.
\par
Planned
\begin{itemize}[noitemsep]
	\item Docking manager + multiple text editors
	\item Improve integration of HlslTools: folding, code completion, ...
	\item C++/CLI project instead of marshalling for ShaderCompiler
\end{itemize}

Backlogged
\begin{itemize}[noitemsep]
\item Audio support: Fast Fourier Transform, Beat Detection as shader input
\item Animation support
\item File formats which support animations
\item Deferred rendering
\item Supporting arrays
\item Supporting the row\_major and column\_major keywords
\item Feature to make scenes which contains multiple shader projects
\item Support for more topologies
\item Parsing of initial values for parameters
\item Blend states
\item And many more...
\end{itemize}

\newpage
\subsection{Workings}
When creating a new shader you can choose which type of shader you want to make, the passes it will need, the topology and set up the rasterizer. Afterwards Shader Box will initialize a new shader project with for each choosen pass a file. The user can always add extra header files which can be used locally so only the project can access it, or set it as a shared header so all the projects can include it. There is one built-in shared header which includes the samplerstates which the engine supports, two cbuffers one for the camera information which changes every frame and one for the per object information (currently only one object per scene). And a reserved slot for a texture, for when creating a post processing shader.
\par
After writing your shader you can click compile or build. Compile will only compile the active shader pass and output the found errors, while build will compile the entire project to precompiled shader files and create a properties panel, for controlling the parameters. The look of the properties panel can be controlled by metadata, using the same syntax as in FX Composer. Finding all this information is done by calling the SyntaxFactory.ParseSyntaxTree(...) method of HLSL Tools on each pass seperatly while passing the shader content. Next Shader Box will go over each SyntaxTree and extract the necessary data for creating the properties, for each found cbuffer a byte array is created which stores the settings which needs to be passed to the GPU. Every variable gets a offset to where it value needs to be stored in the array, in conformance with the HLSL packing needs. Every time the user updates a variable this byte array, which is defined in C\#, goes through a C++/CLI project to the C++ engine.

\newpage
\subsection{Tab: Library}\label{subsec:tab_l}
Panels:
\begin{itemize}
	\item \textbf{Overview}: shows all the created shaders.
	\item \textbf{Shader Properties}: shows the variables which can be passed into the selected shader.
	\item \textbf{Viewport}: for visualizing the selected shader.
\end{itemize}
\scalegraphics{Library.png}{Library tab: with a cubic lens distortion post processing shader selected.}

\newpage
\subsection{Tab: Editor}\label{subsec:tab_e}
Panels:
\begin{itemize}
	\item \textbf{Explorer}: contains all the shader projects you have opened for editing inside the library overview
	\item \textbf{Code view}: a text editor showing the contents of the selected shader file. And an error window showing the found errors when compiling the shader.
	\item \textbf{Shader Properties}: shows the variables which can be passed into the selected shader.
	\item \textbf{Viewport}: for visualizing the selected shader.
\end{itemize}
\scalegraphics{editor.png}{Editor tab: with a silhouette and grease edge rendering geometry shader selected.}

\newpage
\subsection{Tab: Data}\label{subsec:tab_d}
Sub tabs:
\begin{itemize}
	\item \textbf{Images}: Add and remove the image files which can be used as a Texture2D parameter.
	\item \textbf{Models}: The 3D model files between you can choose to display inside the viewport. Currently only Wavefront .obj are supported.
\end{itemize}
\scalegraphics{data.png}{Data tab}

\newpage
\subsection{Sub window: New shader}\label{subsec:tab_d}
Inside the popup window for adding a new shader project you can choose between two shader modes
\begin{itemize}
\item Standard
\end{itemize}
If you want to apply the shader to an object. Requires a vertex and pixel shader, and can have optionally a Hull and Domain shader and/or a geometry shader.

\begin{itemize}
\item Post processing
\end{itemize}
When you want to write a Post processing effect, only pixel shader is required and the other shader types are disabled.
\par
A vertex shader is not required, due to the engine providing this. The engine internally uses a vertex shader utilizing the Fullscreen Triangle Optimization.

Other settings you can set inside the window are:
\begin{itemize}[noitemsep]
\item Topology to be used: Triangle List, Triangle List Adjacency or a 3 Control Point Patch List. Depending on the selected shader stages
\item Cull mode of the rasterizer
\item Fill mode of the rasterizer
\item Name of the shader project
\item Description of the shader project
\end{itemize}
\scalegraphics{options.png}{New shader window: showing the two sub tabs side by side}

\end{document}